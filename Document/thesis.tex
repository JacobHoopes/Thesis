
\documentclass[12pt,twoside]{reedthesis}

\usepackage{graphicx,latexsym} 
\usepackage{amssymb,amsthm,amsmath}
\usepackage{longtable,booktabs,setspace}
\usepackage[hyphens]{url}
\usepackage{rotating}
\usepackage{natbib}
\usepackage{xcolor}
% Comment out the natbib line above and uncomment the following two lines to use the new 
% biblatex-chicago style, for Chicago A. Also make some changes at the end where the 
% bibliography is included. 
%\usepackage{biblatex-chicago}
%\bibliography{thesis}

% \usepackage{times} % other fonts are available like times, bookman, charter, palatino

\title{L-Systems: The Good, the Bad, and the Beautiful}
\author{R Jacob Hoopes}
\date{May 2023}
\division{Mathematics and Natural Sciences}
\advisor{Dylan McNamee}
\department{Computer Science}

\setlength{\parskip}{0pt}
%%
%% End Preamble
%%
%% The fun begins:
\begin{document}

  \maketitle
  \frontmatter % this stuff will be roman-numbered
  \pagestyle{empty} % this removes page numbers from the frontmatter

% Acknowledgements (Acceptable American spelling) are optional
% So are Acknowledgments (proper English spelling)
    \chapter*{Acknowledgements}
	Thanks to my roommates, Henry Wilson, Brent Ellis, and especially Gabe Fish for the late night conversations that led to some of the more interesting choices made and questions raised throughout this document.



    \tableofcontents
% if you want a list of tables, optional
   % \listoftables
% if you want a list of figures, also optional
  %  \listoffigures

% The abstract is not required if you're writing a creative thesis (but aren't they all?)
% If your abstract is longer than a page, there may be a formatting issue.
    \chapter*{Abstract}
	The world is a mess of systems. I hope to establish the context for one system that interests me particularly, and build it up for you so that you may be as interested in it as I have been.
	
    \chapter*{Dedication}
	This thesis is dedicated to the worms in the ground and the trees waving in the wind.

  \mainmatter % here the regular arabic numbering starts
  \pagestyle{fancyplain} % turns page numbering back on

\chapter*{Introduction: The }
	\addcontentsline{toc}{chapter}{Introduction}
	\chaptermark{Introduction}
	\markboth{Introduction}{Introduction}
	
\textcolor{blue}{[Illustration of a simple L-System, perhaps just a fractal tree]}
	
The world of Computer Science can seem like a wilderness, teeming with frightening phrases like Zero-knowledge proofs, NP-Completeness, Finite State Machines, and the Halting Problem, but I hope to open up this world as one worthy of exploration. Specifically exploration at the hands of those from distant disciplines, like Anthropology, Economics, and Psychology, as well as the locals, pure Math, Bio, Chem, and Physics. The tool I will show you and give to you will demonstrate the accessibility of one of the most powerful, subtle, and yet strangely familiar disciplines: Computer Science. These tools have been given the name L-Systems, and their capabilities and beauty can astound.
	
Before we jump in, there are some things that must be said. This is a Computer Science Thesis, and so (essentially by necessity,) there will be code. I would like to keep this Thesis as accessible as possible to those who don’t code, and so I have tried to design it so that anyone with the ability to feel wonder and joy can find something of value between these covers. The code will be accessible on my github page (github.com/JacobHoopes/Thesis) and hosted at at least one other location if possible. The code of the central project will also be accessible in the last pages of the Thesis document itself. However! I have included pictures on every page I could, and I’d like to imagine that they tell enough of a story by themselves that the words are only a lovingly crafted garnish. I suppose we’ll see if I’m right. 
	
\textcolor{blue}{[Illustration of an L-System, something that parallels the first one, and yet does something dramatically different (maybe the same image, just upside down? Inverted somehow?) Maybe just another cool L-System.]}

\section{Motivations}
	
There are several motivations for this thesis, each which I hope will appeal to people from different disciplines with radically different interests. I hope that this thesis will be able to pull on interesting threads in enough of a variety of disciplines that everyone will find something interesting to look at in this short body of work. The primary motivation, and the reason that I am personally so invested in this topic, is the idea of \textit{pattern}. Patterns that I demonstrate with L-Systems and other generative arts appear at different scales throughout life, the world, and our universe. My friends and acquaintances are often much more versed than I am in many of the structures that I reference throughout this thesis, in areas such as biology and chemistry, but also astronomy and physics. I hope to draw in Economists and Anthropologists as well with discussion of patterns that appear in the structures of society and social organization, and which can be illustrated to some extent with the methods that I outline here.

	The second main motivation, one that I try to emphasize at every opportunity, is the pure artistry behind so much of the patterns and methods that I explore. This is the motivation that I imagine might be the most interesting to artists, be they visual artists or otherwise. This may be the reason I was drawn into discussion of L-Systems and their relatives in the first place, so it is as essential a part of this story as the first motivation. 
	
	The third motivation is interested in a philosophical perception of these structures. I am not a trained philosopher, and I would not be surprised to learn if many of the questions that I ask as a part of my exploration are in well-tread ground. It’s possible the questions I ask in this thesis will contribute to the conversations where I imagine they might find a home. Some of the questions I ask address larger issues of the place of machines in the creation of art and the creative process. These questions feel to me to be inseparable from the rest of the work.
	
	I hope that there will be future interest in L-Systems, either as a result of the work that I’ve done here or as part of some other popular exploration of their capabilities. This came to be a motivating project for me for more reasons beyond the three outlined above, and I imagine that someone who sets new eyes on this work could see it in a way I’d never could have foreseen. I go into more detail near the end of the paper about what I believe I have contributed to the conversation around L-Systems and their relatives, as well as what I believe this tool to be capable of with the proper aid.

	
\chapter{What is an L-System?}
    	(What are we talking about? What have I dedicated months of my time and energy to communicating about as best I can? The answer feels complicated after having delved into the depths for a while, but an introduction should be shallow, and so I’ll do that now. L-Systems are recursive replacement system with a unique visual identity. What is a recursive replacement systems? [Perhaps explain in detail, perhaps not - remember, I want to keep this accessible])

—————

This is an L-System:

\textcolor{blue}{[Picture of impressive looking L-system]}

Well, it would be more right to say that this is the \textit{output} of an L-System. L-Systems are a \textit{technique} with which one can describe a complex, changing system with just a few letters and symbols. Some Biologists may take offence, but it may be compared to DNA in how it compresses a potentially infinite complexity into a handful of characters. This chapter will cover a few main ideas. The first, and most central, is a discussion of the mechanics of how they work and how to build them. After that, I’ll go on to describe some of their features, some of their other traits and patterns, and their applications. As part of that last point, I’ll illustrate particularly effective ways these systems can be used, how I’ve used them, (including some early experimentations before the central project,) and end with a lead-in to the central coding portion of this thesis, the L-System Drawer.

There are a couple effective ways I've found to introduce L-Systems to people. I'll go through each of these different strategies to give you some solid footing before we start to start to delve into the more complicated material. 

\section{Understanding through Definitions}
L-Systems have three main parts. The first part is the "start" variable. It is usually just a character, we'll call it ``A" here. The second part is a set of rules that describe how the L-System changes. These are usually written like ``A \rightarrow AB" or ``B \rightarrow AA", where each of these is a different rule. There are different names for what ``A" and ``B" are, but in this thesis I will call them ``Productions". In a similar fashion, each of these rules is called a ``Production Rule". (Any character without a corresponding production rule is therefore not a production. This will be relevant later.) The third part of an L-System is a list of other information that is used when it is being constructed. For most of the L-Systems discussed in this thesis, there are only two pieces of information in this list, a value \theta which is read as the "angle" and a value ``Iterations" which is read as the number of times the system is generated. Alternatives are discussed in Chapter 4.

\section{Understanding through Images}
By using this sort of image to describe these structures, I define the dimensions in which they operate and therefore lose access to some of their possible complexity. I suggest alternative visualization strategies in Chapter 4.
\colortext{blue}{[A series of images that show the progression of an L-System through multiple generations. Each of the images is labelled with a short description that explains the process shown in that image.]}

This set of images is generated with the parameters, start: A; rules: A \rightarrow AB, B \rightarrow AA; \theta: 45\degree. Different values for iterations are displayed in each image. ``A" is drawn as a line going upward and ``B" is drawn as a line going to the right. Each new line starts from where the last line left off.
\colortext{blue}{Image A0}: Iterations = 0, current string: A
\colortext{blue}{Image A1}: Iterations = 1, current string: AB
\colortext{blue}{Image A2}: Iterations = 2, current string: ABAA
\colortext{blue}{Image A3}: Iterations = 3, current string: ABAAABAB


Here is a different set of images with the same start variable and extra information, but alternative production rules. These rules are: A 
\colortext{blue}{Image B0}: Iterations = 0, current 
\colortext{blue}{Image B1}: Iterations = 1;
\colortext{blue}{Image B2}: Iterations = 2;
\colortext{blue}{Image B3}: Iterations = 3;

\section{Understanding through Big Words}
L-Systems are what we in CS would call a \textit{Recursive Replacement structure}. This describes the way in which an L-System makes more of itself. The recursive aspect comes from how 

\chapter{Explaining the Program}



L-Systems are a way to see every stage of the life-cycle of a tree simultaneously. There are limitations to this metaphor, but the freedom of understanding that is gained with the metaphor more than makes up for the rough patches. It’s also accurate especially by the use of the “tree” metaphor. 

\chapter{Contribution}
	A primary goal of this work is to make L-Systems, and by extension, generative art, and even further, Computer Science as a discipline, less frightening. I want people who had no intention of getting involved in the intricacies of 1s and 0s begin to see beauty in the geometry. I want folks who have never used computers used to create things on their own initiative to be amazed and inspired by the possibilities of this program and others like it. I even want my peers, who have been immersed in the depths of CS for years, to be able to approach the discipline which they’ve sunk their whole minds into, with a fresh awe and understanding that the possibilities are as boundless as they’ve dreamed and perhaps even more so. There are grand dreams, to be sure, but computers take the largest of tasks with little more difficulty than the smallest, and art can inspire us to do great things. I hope that this will spark something like that in my readers, though even just using the resources I’ve collected and attach to pursue your own interests is a dreamy enough prospect.
	
	A more humble contribution of this thesis is the raising of awareness of L-Systems and their capabilities. These are tools which I believe to have great power, and I hope that I’ve convinced you that they’re worth further investigation. 
	
	I hope to have used these structures to lower the bound for entry, or at least increase my readers’ enthusiasm for the realm of Computer Science. The field is clearly much more than pretty pictures, but as with many things I feel that a warm welcome will help those who were once strangers come into this place as their home. These pictures are my attempt at a warm welcome. 

\section{Related Work}


\chapter{Next Steps}

\section{Generalizing L-Systems}


\chapter*{Conclusion}
         \addcontentsline{toc}{chapter}{Conclusion}
	\chaptermark{Conclusion}
	\markboth{Conclusion}{Conclusion}
	\setcounter{chapter}{4}
	\setcounter{section}{0}

    \appendix
      \chapter{Extra Images}
      
      \chapter{Generative Art Resources}

\backmatter 
\nocite{*}
\bibliographystyle{APA/apa-good}
\bibliography{thesis}
 
\end{document}
